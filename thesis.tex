\documentclass[12pt,a4paper]{report}

% English
\usepackage[english]{babel} % English language setting
\usepackage[utf8]{inputenc} % Unicode text
\usepackage[T1]{fontenc} % German 'Umlaute'
\usepackage{textcomp} % Euro
\usepackage[hyphens]{url}
\usepackage{amsmath} % Advanced mathematical notation
\usepackage{amssymb} % Symbols
\usepackage{emptypage} % Empty pages are now actually empty

% Fonts, with all the options
\usepackage{mathpazo}
\usepackage[scaled=.95]{helvet}
\usepackage{courier}
\usepackage{microtype}

% Images and listings
\usepackage{graphicx} % images
\usepackage{subfig} % sub-figures
\usepackage{wrapfig} % wrapping figures
\usepackage{listings} % better source code listings
\usepackage{tabularx} % better tables
\usepackage{hhline}
\usepackage{enumitem}
\usepackage{array}
\usepackage[colorinlistoftodos]{todonotes}
\usepackage{booktabs}
\usepackage{url}
\usepackage{breakurl}
\graphicspath{ {figures/} }

% Page layout
\usepackage[paper=a4paper,width=14cm,left=35mm,height=22cm]{geometry}
\usepackage{setspace}
\usepackage[htt]{hyphenat}
\usepackage{sectsty}
\linespread{1.5}
\subsubsectionfont{\large}

% Page markers
\newcommand{\phv}{\fontfamily{phv}\fontseries{m}\fontsize{10}{12}\selectfont}
\usepackage{fancyhdr} % nicer header and footer
\pagestyle{fancy}
\renewcommand{\chaptermark}[1]{\markboth{#1}{}}
\fancyhead[L]{\phv \nouppercase{\leftmark}}
\fancyhead[R]{\phv \thepage}
% rather not use anything for the footer
\fancyfoot[C]{\ } % no page count in the bottom
%\fancyfoot[R]{\textsf{\small Software Engineering MS}}

% Share the sources
\usepackage{bibtopic}

% Special packages
\usepackage{epigraph}
\setlength{\epigraphrule}{0pt} % no divider
\usepackage{csquotes}

% Some extra styles
\usepackage{soul}
\newcommand*\strikethrough{\st}

% Hyperlink everything
\usepackage{hyperref}
\hypersetup{
    bookmarks=true,
    colorlinks=true,
    linkcolor=black,
    citecolor=black,
    filecolor=black,
    urlcolor=black,
}
\urlstyle{same}

\newcommand*\rot{\rotatebox{90}}
\newcolumntype{Y}{>{\centering\arraybackslash}X}

% Wikipedia-style "citation needed" macro
\newcommand{\cn}[1][]{\textsuperscript{\color{red} ~[citation needed]~}}

% Code markup
\newcommand{\code}[1]{\texttt{#1}}


\newcommand{\studyprogramme}{Software Engineering}
\newcommand{\degreetype}{Master of Science}
\newcommand{\thesistitle}{
Toward systematic decomposition of monolithic software into microservices
}
\newcommand{\thesissubtitle}{
SUBTITLE
}
\newcommand{\thesisauthor}{Jakob L{\"o}hnertz}
\newcommand{\thesisdate}{31/08/2019}
\newcommand{\thesislocation}{Amsterdam}
\newcommand{\firstmarker}{Dr.\ Ana Oprescu}
\newcommand{\secondmarker}{MSc.\ Stephan Schroevers}
\newcommand{\hostorganization}{Picnic B.V.}

\begin{document}



\begin{titlepage}
  \begin{center}
    \parbox[c]{\textwidth}{\uva}
    \\[2.5cm]
    {\LARGE Thesis} \\[0.35cm]
    {\begin{spacing}{1} to obtain the academic degree \\[0.35cm] \end{spacing}}
    {\begin{spacing}{1} \LARGE \degreetype \\[1.5cm] \end{spacing}}
    \rule{\textwidth}{1pt} \\[0.55cm]
    {\begin{spacing}{1.15} \huge \bfseries \thesistitle \\[0.35cm] \end{spacing}}
    {\begin{spacing}{1.15} \bfseries \thesissubtitle \\[0.60cm] \end{spacing}}
    \rule{\textwidth}{1pt}
    \\[1.5cm]
    \begin{tabular}{l l}
      Author & \thesisauthor \\
      Research supervisor & \firstmarker \\
      Host supervisor & \secondmarker \\
      Host organization & \hostorganization \\
    \end{tabular}
  \end{center}
\end{titlepage}
\cleardoublepage

\thispagestyle{empty}
\section*{Affidavit}
I hereby declare that I have
\begin{itemize}
\item independently composed this thesis,
\item not used any other sources than those indicated,
\item identified as such, either verbatim or according to the content, any parts taken from external works, pictorial representations and the like,
\item not made use of any unauthorized external help.
\end{itemize}

\vspace{6em}
\noindent\begin{tabular}{p{0.37\textwidth}p{0.56\textwidth}}
\thesislocation, \thesisdate  & \rule{0.56\textwidth}{0.5pt}\\
              & \makebox[1cm]{\ } \thesisauthor
\end{tabular}

\vfill

\cleardoublepage

\raggedbottom


\thispagestyle{empty}
\section*{Acknowledgments}
First and foremost, I want to thank my family and my friends here in Amsterdam and back at home in Germany.
They provided support, proofreading, and fun times during this entire summer.
Specific thanks go to my girlfriend Núria Bruch Tàrrega as well as to my two best friends Jean Marcel Petry and Eric Richter who always have an open ear for me, no less during the thesis period.
Additionally, I specifically want to thank Christian Stuart, Sander Meester, and Changhun Yun
who aided me relentlessly when I got sick toward the end of the writing period for multiple weeks.
In the same vein, my family massively helped me in getting healthy again.
Their daily support comforted me a lot and although I could not turn the thesis in, in the end of August
as originally planned, they sped up the time it took me to get back on track immensely.
Moreover, I want to thank Moustafa Selim who took on the burden to proofread my entire thesis.

Furthermore, I want to thank Ana Oprescu, my academic supervisor, who guided me through the process
and always gave great recommendations that contributed impactful changes to my thesis,
specifically regarding the structure --- I am much happier with it after implementing
her many tips than before. Apart from that, Ana is always available for problems and does a great
job managing the entire thesis procedure of our master program. Moreover, I was humbled when she
offered to co-author a paper regarding my thesis topic.

Finally, I want to thank the entire Picnic company that hosted my thesis and specifically my awesome
company supervisor Stephan Schroevers who always carved extra minutes out of his incredibly busy schedule
to sit down with me to discuss my progress or arising questions on the spot. Moreover, our weekly meetings
sparked so many ideas that I then implemented, I cannot count them anymore. He is the main reason that
I initially accepted this thesis project and I am happy that I had him as my daily supervisor and hope
to now experience many interesting work hours together with him.
I want to thank Daniel Gebler for believing in me and bringing me into this great company,
Elise Baeriswyl for always having an open ear, being a great conversation partner,
and handling day to day problems as well as the entire HR team of Picnic,
and Philip Leonard for giving good tips regarding the thesis and for being a great friend.
Additionally, thanks go to my coworkers Matthijs IJkema, Jelmer Borst, Arthur Degon-Knifton, Rohit Sharma,
and Nathan Kooij, who welcomed me into their team for the entirety of the summer to sit with them
and that gave daily feedback on my progress.
I did not take this for granted but I was very happy that you all took me.
Lastly, one of the biggest thanks go to Jean de Leeuw and Katarina Lang, my fellow thesis students here
at Picnic, who I had many good discussions with regarding our progress while they eventually also became
good friends.
Last but not least, I want to thank Sjoerd Cranen, Max Sumrall, and Matthijs IJkema for believing in
me and offering me a spot to stay in this awesome company even after my thesis.

\vfill

\cleardoublepage

\raggedbottom




\begin{abstract}

ABSTRACT

\end{abstract}



\tableofcontents



\chapter{Introduction} \label{chap:introduction}

Software engineering develops faster than ever, the amount of novelties in
platforms, languages, operational strategies and generally the terminology
around it gets larger and larger every day.
One of the terms that gained a lot of traction in recent years is \textit{microservices}.
Many engineers talk about their benefits and challenges but the general notion
is positive and the buzz is undeniable.
At the end of the day, microservices are nothing more than an architectural style
for designing back-end software.
For those that actually take the step to build and deploy such an application
landscape, the journey is not always straightforward.
The idea is still evolving and mandates many new strategies and concepts
that developers and operations have to deal with
when choosing to devote their business logic to leverage microservices.

There are many good reasons to employ this pattern nowadays specifically
when dealing with ever larger growing monolithic applications that try to
encapsulate their respective business logic without drowning the
developers and engineers working on it in seemingly endless complexity
that a single human cannot even fully comprehend anymore at one point.

For many companies, this oftentimes is the status quo and if the decision is
made to move to a microservices-based architecture, one of the first
challenges, the decomposition, is already difficult to begin with:\\
Defining where within the existing application one service should end and the
next one should begin --- i.e. detecting latent boundaries that partition
the monolith into many smaller pieces, is a very manual and tedious task.

Thus, this thesis develops an algorithmic methodology that can assist
a human software engineer in performing the aforementioned decomposition.


\section{Problem analysis}

Generally speaking, there are two ways to put a microservices-based
software architecture into place. The greenfield approach, meaning that
microservices are directly getting built without a preexisting application,
as well as the transformation of an existing monolithic software solution
into independent microservices \cite{fowler-break-monolith}.
However, this thesis focuses solely on the latter approach for a variety of reasons.

Firstly, our work leverages static and dynamic program analysis for solving
the posed task. Obviously, this is not possible without an already running solution.

Secondly, while there is already some research on this topic \cite{fritzsch2018monolith},
which is discussed in the \nameref{chap:related} chapter, the notion
of utilizing the already existing and running application is still underrepresented.

Thirdly, as Martin Fowler remarks, microservices are especially feasible
with a mature workflow, domain model, operational model, etc. already
in place \cite{fowler-monolith-first} \cite{fowler-microservices-tradeoffs}
which favors non-greenfield approaches as well.

The task of getting the boundaries of each microservice right is hard,
especially when the teams working on the existing software
are caught up in their mental models of it \cite{latoza2006maintaining},
as Fritzsch et al. confirm: \textit{"Extracting a domain model from
an application's code base can be a significant challenge. If incorrectly applied,
it can lead to architectures that combine the drawbacks of both styles,
monolithic architectures and Microservices."}\cite{fritzsch2018monolith}
To prevent such an outcome, we propose a visual approach to assist software engineers
and architects in decomposing microservices out of an existing monolithic code base.
This approach has the advantage that the mental models regarding the existing
solution might be broken apart partially to offer new perspectives onto how
the current software works, which was already remarked as a positive side effect
of an assisted microservices decomposition approach \cite{gysel2016service}.
Thus, the engineers and architects can be assisted in the first step,
which is likely also the most difficult and time-consuming one due to the possible
complexity of the system \cite{fritzsch2018monolith} \cite{france2007model},
when undertaking the operation of transforming their existing
monolithic solution into microservices.

Additionally, we are certain that the actual decisions should still be made
by seasoned engineers and architects which is why a visual solution
was determined as the best option to actually support them in their
decision making while not taking the decision making away from them
(i.e. compared to an automated decomposition approach).
As a consequence, everything mentioned in this section reflects itself in
the following research questions of this thesis project.


\section{Research methods} % TODO: Validate this section




\section{Research questions}




\chapter{Related work} \label{chap:related}




\section{Literature survey} \label{subsect:literature-survey}




\section{Topic relations} \label{subsect:topic-relations}




\subsection{Analyzed inputs}




\subsection{Graph clustering}




\subsection{Evaluative metrics}






\chapter{Background} \label{chap:background}





\section{Software architectures} \label{sect:background-architecture}




\subsection{Monolithic}



\subsubsection{Advantages}



\subsubsection{Disadvantages}



\subsection{Microservices-based}



\subsubsection{Advantages}



\subsubsection{Disadvantages}




\section{Software analysis} \label{sect:background-program-analysis}

\subsection{Static program analysis}

\subsection{Dynamic program analysis}

\section{Graph clustering} \label{sect:background-graph-clustering}

\subsection{Renowned algorithms}

\subsection{Available metrics}




\chapter{Mapping microservice requirements to software quality metrics} \label{chap:rationale}

\section{Pillars of microservices}

\section{Selecting and extracting software quality metrics}




\chapter{Extracting coupling information from software}

\section{Dynamic coupling}

\subsection{Profiling-based approach}

\subsection{Instrumentation-based approach}

\section{Semantic coupling}

\subsection{Natural language processing on source code}

\section{Logical coupling}

\subsection{Mining version control system data}




\chapter{Representing software as weighted graphs}

\section{Building a weighted graph for each input dimension}

\section{Merging input dimensional graphs into combined one}




\chapter{Creating microservice recommendations}

\section{Clustering combined weighted graph}




\chapter{Calculating metrics on microservice recommendations}

\section{Input fidelity}

\section{General clustering quality}

\section{Coupling modularity}




\chapter{Visualizing microservice recommendations}

\section{Network-based view}

\section{Tree-based view}

\section{Metrics view}




\chapter{Implementation}




\chapter{Results}

\section{Experimental setup}




\chapter{Discussion} \label{chap:discussion}

\section{Threats to validity} \label{sect:threats-to-validity}

\section{Future work} \label{sect:future-work}




\chapter{Conclusion} \label{chap:conclusion}






% Source lists
\newpage
\addcontentsline{toc}{chapter}{List of Figures}
\listoffigures
\newpage
\addcontentsline{toc}{chapter}{List of Tables}
\listoftables
\newpage
\addcontentsline{toc}{chapter}{Bibliography}
% Separate the sources with 'bibtopic'
\bibliographystyle{plain}
\begin{btSect}{references}
\section*{\huge{References}}
\btPrintCited
\end{btSect}
\begin{btSect}{online}
\section*{\huge{Online Sources}}
\btPrintCited
\end{btSect}

\end{document}
